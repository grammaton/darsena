\documentclass[a4paper,11pt]{article}

\usepackage[italian]{babel}
\usepackage[utf8]{inputenc}
\usepackage{amsmath}
\usepackage{amssymb}
\usepackage{graphicx}
\usepackage[hidelinks]{hyperref}
\usepackage{grffile}
\usepackage{braket}

% Personalizzazione del documento
\title{Pensare Tetraedrico Oggi}
\author{Giuseppe Silvi}
\date{\today}

\begin{document}
\maketitle

\begin{center}
\rule{3in}{0.4pt}
\end{center}
title: ``Percorso Storico''
layout: page
parent: ``Pensare Tetraedrico Oggi''
nav\_order: 1
last\_modified\_date: 2025-02-04 12:26:00 CET
---

\subsection{Percorso storico}\hypertarget{percorso-storico}{}\label{percorso-storico}

L'uomo che più di tutti avrebbe sorriso alle evoluzioni delle mie
stravaganti idee sullo spazio sonoro sarebbe stato \emph{Piero Schiavoni}.
Avrebbe sorriso, consapevole di esserne stato la causa.

\textbackslash{}begin\{figure\}{[}htbp{]}
  \textbackslash{}centering
  \textbackslash{}includegraphics{[}width=0.8\textbackslash{}textwidth{]}\{/Users/giuseppesilvi/Documents/github/gs/darsena/\}
  \textbackslash{}caption\{no-alignment\}
\textbackslash{}end\{figure\}

\emph{Piero} è stato docente di \emph{Elettroacustica} presso il \emph{Conservatorio S.
Cecilia} di Roma dal 2002 al 2012. È lì che l'ho conosciuto, nel 2007.
Le sue lezioni erano così belle che la loro bellezza ti raggiungeva
prima ancora di iniziare i suoi corsi. Gli altri studenti ne parlavano
come di un personaggio mitologico, dalla sapienza teorica seconda solo
all'appassionata pratica tecnica. Si creò tra noi, in un tempo piccolo,
un rapporto speciale. Il nostro grado comune era l'\emph{ambisonic}, anello
che lui stesso aveva forgiato. Quando Piero iniziò a parlarci di
\emph{ambisonic} fu per noi un momento di alta formazione perché di fatto
partecipavamo, nell'unica realtà del presente, alle sue ricerche. La
materia di cui lui ci narrava le gesta era argilla fresca tra le sue
mani. Iniziammo a lavorare attivamente alle teorie di \emph{Michael
Gerzon} nel 2008, mettendo in pratica e simulando molti dei test di
ascolto che egli suggeriva. Era questo il metodo di Piero, ascoltare per
capire e, poi, credere. Un metodo didattico che vide il suo picco di
bellezza durante le giornate di registrazione comparata tra
configurazioni microfoniche \emph{stereo} o \emph{surround}.

\textbackslash{}begin\{figure\}{[}htbp{]}
  \textbackslash{}centering
  \textbackslash{}includegraphics{[}width=0.8\textbackslash{}textwidth{]}\{/Users/giuseppesilvi/Documents/github/gs/darsena/\}
  \textbackslash{}caption\{no-alignment\}
\textbackslash{}end\{figure\}

Nel 2008 nacque \emph{EMUFest}{[}\^{}4{]}. \emph{EMUFest} fu il prototipo di didattica
aumentata, il \emph{fare} della scuola romana portato oltre ogni aspettativa,
che trascinò noi, gli studenti, dentro la musica e non verso la musica.
Con circa cento brani l'anno provenienti da tutto il mondo e suonati in
concerti da studenti, docenti e professionisti esterni è stata, a mio
avviso, una rarissima forma di didattica reale, possibile e necessaria,
l'unica che questo contesto artistico e disciplinare dovrebbe avere. Nel
2009, per la seconda edizione del festival, realizzai per Piero il
software che ci avrebbe permesso di utilizzare la tecnologia
\emph{ambisonic} nella sala da concerto del \emph{Conservatorio S. Cecilia}.

Del 2009 fu anche la prima sessione di registrazione comparata di
configurazioni stereofoniche a cui ho partecipato{[}\^{}5{]}. Nel 2010 facemmo
visita con una piccola delegazione di studenti (con me, Federico Scalas,
Leonardo Zaccone, Simone Pappalardo e lo stesso Piero) presso la \emph{Casa
del Suono} di Parma dove ascoltammo le parole di \emph{Fons Adriaensen}, i
suoni della \emph{Sala Bianca} in \emph{WFS} ed alcune registrazioni fatte da noi
in \emph{ambisonic}. L'incontro con \emph{Fons} portò ad un livello superiore il
nostro lavoro con l'\emph{ambisonic}, con la consapevolezza che la strada che
stavamo percorrendo era ormai la nostra.

Non un ambiente istituzionale dal rigore scientifico, ma un laboratorio
dove la mente e le braccia lavorano rigorosamente al servizio della
musica. Da qui viene il mio pensiero ed il mio modo di fare. Da qui
posso partire con serenità e spiegare la mia attuale visione della
musica.



\subsubsection{Rapporto di ricerca \{\#rapporto-di-ricerca .unnumbered\}}\hypertarget{rapporto-di-ricerca-rapporto-di-ricerca-unnumbered}{}\label{rapporto-di-ricerca-rapporto-di-ricerca-unnumbered}

\begin{quote}
{[}I{]} L'arte musicale consiste nella conoscenza profonda acquisita con
l'esperienza, della modulazione\footnote{Per il significato assunto dal termine \emph{modulatio} (da \emph{modus},
misura) nel linguaggio musicologico medievale cfr. AGOSTINO, \emph{De
musica}, I, 2, 2 {[}\ldots{}{]} abilità grazie alla quale avviene che un
qualcosa si muova in maniera conveniente.} ed ha il proprio fondamento nel
suono e nel canto. Il termine \emph{musica} trae origine dal nome delle
Muse, così chiamate \textbf{parole greche}, cioè \emph{dall'atto del ricercare},
poiché gli antichi ritenevano che fosse necessario il loro aiuto al
momento di ricercare la forza espressiva da infondere nei carmi e la
giusta modulazione della voce. {[}2{]} La voce delle muse, in quanto
oggetto dei sensi, o svanisce con il trascorrere del tempo oppure si
fissa nella memoria: proprio per questo, dunque, le stesse muse sono
state immaginate dai poeti figlie di Memoria e Giove. I suoni,
infatti, non sono trattenuti dall'uomo attraverso la facoltà della
memoria, muoiono, poiché non possono essere fissati mediante la
scrittura.
\end{quote}

Il primo pensiero sull'altoparlante tetraedrico risale al 2009. Ci fu un
rapido scambio di opinioni con Piero che mi rispose \emph{``nemmeno *Michael
Gerzon}  ha mai osato tanto''\emph{. Era chiaramente un affettuoso modo di
incitare il mio lavoro. Nonostante ciò non andai mai oltre un misero
bozzetto, qualche descrizione dei principi costruttivi e funzionali. Poi
arrivò l'anno del cambiamento, il 2013. L'anno si aprì con la morte di
Piero e si chiuse con il pensionamento di *Giorgio Nottoli}, il mio
maestro di sempre. Giorgio mi portò alla musica, al conservatorio, a lui
devo l'infinito presente musicale. Come solo la vita sa fare, con il suo
ciclico mutevole rigenerare ci fu subito una nuova persona ad
appassionarsi alla mia ricerca. \emph{Michelangelo Lupone}, poco dopo essere
salito in cattedra a Roma succedendo \emph{Giorgio Nottoli}, rese possibile,
con poche parole e qualche prezioso consiglio, la ripresa dei lavori di
prototipizzazione con un preciso percorso da seguire. Le lezioni del
biennio magistrale erano esaltanti. Eravamo soltanto due studenti e come
me anche Massimo Massimi stava lavorando ad un progetto elettroacustico
per la diffusione da concerto denominato \emph{Sicomoro}. Il confronto era
continuo, quotidiano. Arrivai a produrre il mio primo esemplare
funzionante nell'ottobre 2014. Contemporaneamente formulai una tecnica
microfonica associata al diffusore, ispirata a quella di \emph{Michael
Gerzon}  che chiamai \emph{TETRAREC}.

Come non lo era quella di \emph{Michael Gerzon}, anche questa non è una
ricerca focalizzata esclusivamente su tecniche di diffusione sonora
nello spazio. C'è un problema di fondo che emerge e collega queste
ricerche e quelle di chiunque altro si sieda in un contesto in cui ci
siano microfoni e diffusori.

È un problema di fondo, latente, da decenni, fin dal principio delle
tecniche associate al suono riprodotto. È presente anche nel brevetto
sulla stereofonia di Alan Blumlein del 1931-33. Si tratta del prezioso
equilibrio acustico instaurato tra soggetti acustici e di quello
difficilmente ottenibile tra questi ed i sistemi elettroacustici.
::::

\section{Letture}\hypertarget{letture}{}\label{letture}

\subsection{Esposizioni Elettroacustiche}\hypertarget{esposizioni-elettroacustiche}{}\label{esposizioni-elettroacustiche}

{[}{]}\{\#c:ee label=''c:ee''\}

\begin{quote}
\ldots{}il dio gli offerse\textbackslash{}
di scegliersi un premio -- una scelta dannosa,\textbackslash{}
perché Mida ne usò malissimo e disse: ``Fa' in modo\textbackslash{}
che tutto quello che tocco, si muti in fulvo oro''\textbackslash{}
Bacco assentì, concedendo un dono nocivo\textbackslash{}
e rammaricandosi che non avesse scelto di meglio.\textbackslash{}
Il re berecinzio andò via, tutto contento del suo malanno\textbackslash{}
e cominciò, toccando qua e là, a verificare la promessa del dio.\textbackslash{}
A stento trattiene le sue speranze, immaginandosi un mondo\textbackslash{}
tutto d'oro: ma mentre esulta, i servi preparano\textbackslash{}
la mensa imbandita, con pane tostato.\textbackslash{}
Ma adesso, appena la sua mano toccava i doni di Cerere,\textbackslash{}
i doni di Cerere si irrigidivano; \ldots{}\textbackslash{}
Attonito per la stranezza dell'inconveniente, povero e ricco,\textbackslash{}
cerca di sfuggire alla sua ricchezza, odia quello che aveva chiesto.\textbackslash{}
\ldots{}``Perdonami, padre Bacco, ho sbagliato,\textbackslash{}
ma abbi pietà, ti prego, toglimi a questo bellissimo male!'' \footnote{Ovidio, Metamorfosi XI, 100-138}.
\end{quote}

Con il paradosso per cui il dono di Bacco aveva reso il re Mida ricco ed
allo stesso tempo così povero da morire di fame ci si proietta diretti
nel cuore della questione straordinaria di cui è l'elettroacustica, come
dire, ricchi di un dono fantastico come l'ascolto, nella sua
stratificazione fisica e psicologica di udito e pensiero, ricchi di
competenze tecnologiche e scientifiche, di complessi strumenti di
produzione, analisi ed elaborazione, poveri ed incapaci di comprenderne
il significato ed i meccanismi più semplici.

Tutto suona, il mondo è oro per le nostre orecchie. Ricchi ma poveri.
Dotati ma sordi. Così capaci di meraviglie tecnologiche, con mani fatate
mutiamo in oro anche le peggiori schifezze acustiche, per poi essere
completamente incapaci di ascoltarle, comprenderle e giudicarle.

\begin{quote}
Il re berecinzio andò via, tutto contento del suo malanno\textbackslash{}
e cominciò, toccando qua e là, a verificare la promessa del dio\footnote{Ovidio, Metamorfosi XI, 100-138}.
\end{quote}

Così siamo, giratori di pomelli per prodotti luccicanti, ignari del
malanno che portiamo. Usiamo parole fuori dal loro contesto e, con le
parole rubate, designificate, indichiamo oggetti e processi senza
apprezzarne il valore reale, indicando spesso il vuoto. Mastering come
andare dal barbiere, compressione come se avessimo chili di troppo. Di
esempi come questi ne avremmo per giorni e giorni di chiacchiere ma
generalmente si parla poco di visione, con le orecchie, ascolto
compartecipato di orecchie e cervello, di messa in scena dei suoni.

\begin{quote}
\ldots{}``Per non rimanere invischiato nell'oro\textbackslash{}
male desiderato, va' al fiume vicino a Sardi,\textbackslash{}
e cammina sul monte, risalendo le acque,\textbackslash{}
finché arriverai alla sorgente del fiume,\textbackslash{}
e là metti il capo dove è più forte il getto\textbackslash{}
della fonte, e lava insieme il corpo e la colpa''\footnote{Ovidio, Metamorfosi XI, 100-138}.
\end{quote}

Così Bacco ci indica il percorso per lavare le nostre colpe: dobbiamo
lasciare i pomelli e mettere il capo nella sorgente, dove è più forte il
getto. Di ronte al bivio potremmo rassegnarci nel dire che la sorgente,
elettroacustica, è in un momento storico, in un luogo, a valle del
fluido percorso del tempo, che forse abbiamo perso. Oppure potremmo
accettare semplicemnte che la sorgente è nelle parole con cui definiamo
i conncetti e, tornando ad esse, sotto il peso del loro forte getto, ci
laveremo dai nostri malanni.

Mi è capitato più volte, di dover definire la \emph{Stereofonia} ed osservare
reazioni più o meno in accordo, come se si possano avere opinioni in
merito ad una definizione, come se possano esistere diverse sfumature di
\emph{Stereofonia}. È difficile accettarlo, ma generalmente si possono
contemplare opinioni su definizioni linguistiche, anche contrastanti,
perché ci dilettiamo nel fare confusione con le parole.

Il primo passo necessario verso la comprensione del concetto di
\emph{Stereofonia}, per di arrivare alle tecniche ed alle tecnologie
elettroacustiche che la rendono possibile, è stabilire attraverso
l'etimologia del termine, e dei termini ad esso collegati, una base
concettuale solida. \emph{Stereo}{[}{]}\{\#stereo label=''stereo''\}, dal greco
\emph{Stereós}, significa \emph{solido}. Non un numero, non una configurazione ma
un aggettivo qualitativo. Nel dizionario inglese Oxford: \emph{Solid, firm
and stable in shape. Having three dimensions}. Solido, \emph{solid}, dalla
radice latina di \emph{Solidus, Sollus}, intero. Con la parola \emph{Stereofonia}
dovremmo quindi descrivere una condizione nella quale \emph{phonē}, sempre
dal greco, \emph{suono}, la \emph{voce}, arrivi all'ascoltatore solida, integra,
ferma e stabile nella sua forma (sonora) multi dimensionale, intera.

:::: wrapfigure
O0.48

::: center
\textbackslash{}begin\{figure\}{[}htbp{]}
  \textbackslash{}centering
  \textbackslash{}includegraphics{[}width=0.8\textbackslash{}textwidth{]}\{/Users/giuseppesilvi/Documents/github/gs/darsena/images/disegni/figb.png\}
  \textbackslash{}caption\{image\}
\textbackslash{}end\{figure\}
\{width=''45\%''\}
:::

{[}{]}\{\#ee:figb label=''ee:figb''\}
::::

Indispensabile alla comprensione è anche la descrizione del concetto di
\emph{mono}, nomignolo di \emph{monofonico}, espresso nel legame tra \emph{monos} e
\emph{phonē}: una voce, \emph{one voice}, \emph{alone}, sola. La stessa parola usata
nella descrizione del canto gregoriano ad una voce, successivamente
evolutasi nella \emph{polifonia} (dal greco \emph{poluphōnia}, da \emph{polu}, molte e
\emph{phonē}, voci). Quindi la dicotomia, se proprio deve essercene una, tra
monofonia e stereofonia semplicemente non esiste. L'estensione del
concetto di monofonia, nel suo eventuale opposto, è polifonia. La
stereofonia è semplicemente un concetto altro.

Con la parola stereofonia dovremmo descrivere anche la condizione in
base alla quale il suono arrivi \emph{solido} all'ascoltatore, \emph{intero, fermo
e stabile} nella sua forma sonora multidimensionale originaria,
attraverso la riproduzione elettroacustica, attraverso la trasmissione e
la diffusione mediante altoparlanti, con un numero qualsiasi, o
necessario, di canali. In questo caso, facendo riferimento alla
definizione dell'aggettivo \emph{stereofònico}, ci si apre alle tecniche ed
alle tecnologie che hanno reso possibile la trasmissione, la
riproduzione e la diffusione del suono in stereofonia. Un aggettivo che
dovrebbe essere usato con cautela, nella circostanza in cui la
riproduzione dei suoni avvenga in modo che l'ascoltatore abbia
l'impressione di trovarsi nello spazio sonoro originale o, nel caso non
ve ne sia, per sorgenti di natura non acustica, che restituiscano
informazioni tali da descrivere una correlazione al sistema percettivo,
simile a quella suggerita da sorgenti acustiche.

Condizioni di ascolto stereofoniche, dal latino \emph{auscultare}, prestare
attenzione a qualcosa in quanto oggetto o motivo di informazione, nel
caso specifico, informazioni di stereofonia.

\emph{Una voce in una piccola stanza riverberante è una condizione d'ascolto
che rispetti queste qualità?}

Prima di approfondire questioni di propagazione e percezione del suono,
val la pena dedicare un tempo alla letteratura specializzata.

\begin{quote}
When recording music considerable trouble is experienced with the
unpleasant effects produced by echoes wich in the normal way would not
be noticed by anyone listening in the room in which the performance is
taking place. An observer in the room is listening with two ears, so
that echoes reach him with the directional significance which he
associates with the music performed in such room. He, therefore,
discounts these echoes and psychologically focuses his attention on
the source of the sound\footnote{Quando si registra musica acustica, si riscontrano notevoli
problemi a causa degli effetti indesiderati prodotti dalle
riflessioni acustiche dell'ambiente, che nell'ascolto normale non
vengono notati dagli ascoltatori nella stanza in cui si svolge
l'esibizione. L'ascoltatore, nella stanza, ascolta attraverso le due
orecchie, le riflessioni lo raggiungano con il significato
direzionale che associa alla musica eseguita in quella stanza.
Pertanto, elimina dal messaggio le informazioni delle riflessioni e
focalizza psicologicamente la sua attenzione sulla sorgente sonora.}.
\end{quote}

Richiesta di brevetto numero 394325 del 14 dicembre 1931, accettazione
del 14 giugno 1933. \emph{Alan Dower Blumlein}.

:::: wrapfigure
O0.48

::: center
\textbackslash{}begin\{figure\}{[}htbp{]}
  \textbackslash{}centering
  \textbackslash{}includegraphics{[}width=0.8\textbackslash{}textwidth{]}\{/Users/giuseppesilvi/Documents/github/gs/darsena/images/disegni/figc.png\}
  \textbackslash{}caption\{image\}
\textbackslash{}end\{figure\}
\{width=''45\%''\}
:::

{[}{]}\{\#ee:figc label=''ee:figc''\}
::::

La risposta alla domanda \emph{una voce in una piccola stanza riverberante è
una condizione d'ascolto che rispetti queste qualità?} è, in funzione di
quanto appena letto, chiaramente affermativa. Anche con un solo oggetto
sonoro, una sola voce, in una piccola stanza, siamo in presenza di un
fenomeno acustico stereofonico. Almeno così dice Blumlein, papà della
stereofonia, nel brevetto in cui ne rende i concetti fondamentali,
solidi, stabili nel tempo e nello spazio delle parole, nel brevetto
tecnologico che stabilisce il \emph{point break} dell'elettroacustica, per il
resto dell'umanità.

Una voce nello spazio di una stanzetta si dirige, con una sua direzione,
verso un punto e contemporaneamente, con meno direzionalità,
lateralemente, raggiunge il resto della stanza. Questo meccanismo ha a
che fare con la forma sonora di una voce, prima ancora che con la forma
architettonica della piccola stanza. Possiamo immaginare la forma sonora
come un'armatura attorno al nostro oggetto sonoro, un'armatura fatta di
fittissime molecole in vibrazione. Ogni suono ha una sua veste plastica.
Dicendo ottavino e poi contrabbasso voi avrete già collegato tutto ciò
che vi serve per vederli, sentirli, ed ora, volendo, vestirli della loro
forma sonora. Questa si staglia nello spazio circostante e si espande e
si muove all'interno di uno spazio e ne viene modellata come una massa
morbida all'interno di un contenitore. Qui iniziano i fenomeni di
riflessione e la forma si cristallizza assumendo caratteristiche in
funzione dello spazio e, quindi, del tempo. L'ascoltatore che partecipa
a questo evento vede una persona solida parlare nello spazio di una
stanza e sente la forma solida della voce provenire dalla sua bocca e
contemporaneamente, quindi subito dopo, dalla stanza sotto forma di
riflessioni.

Una voce che attraverso la sua forma acustica riempia uno spazio
acustico è un'esperienza d'ascolto stereofonica. Non è ancora giunto il
momento di interrompere la discussione dicendo: ``ma come, non servono
due diffusori?'' Non ancora, il problema è più complesso. È importante
sottolineare che la stereofonia, l'ascolto stereofonico, è una qualità
dell'ascolto che si può osservare in determinate circostanze e che
richiede necessariamente il lavoro concertato delle due orecchie. Un
ascolto stereofonico è quindi possibile solo in coincidenza con un
ascolto \emph{binaurale}, ovvero effettuato con entrambe le orecchie.

\begin{quote}
When the music is reproduced through a single channel the echoes
arrive from the same direction as the direct sound so that confusion
result\footnote{Quando la musica viene riprodotta attraverso un singolo canale,
gli echi arrivano dalla stessa direzione del suono diretto in modo
tale da creare confusione.}.
\end{quote}

Qui si sviluppa tutta la questione, un solo diffusore non è in grado di
rappresentare la solidità originaria, la forma sonora dell'oggetto
acustico originario, il suo rapporto con lo spazio che lo ha modellato.
Per comprendere meglio ogni possibile questione legata alla diffusione
sonora, mediante dispositivi elettroacustici ci vorrebbe un minimo di
tempo speso nella sperimentazione con lo strumento altoparlante. Perché
di questo si parla, di uno strumento tecnico, tecnologico, musicale e
profesisonale.

\begin{quote}
\ldots{}il solo Mida\textbackslash{}
lo criticò e disse che era un'ingiustizia.\textbackslash{}
Il dio di Delo non sopportò che le sue orecchie\textbackslash{}
stolide conservassero figura umana,\textbackslash{}
gliele tirò e allungò, le cosparse di pelame grigio,\textbackslash{}
le rese instabili alla base, che potessero muoversi.\textbackslash{}
Il resto è di uomo; la condanna riguarda una sola parte\textbackslash{}
del corpo -- porta le orecchie del tardo asinello.
\end{quote}

La parabola bacchiana si conclude. Noi, che abbiamo osservato da vicino
la maledizione di Bacco, non possiamo più tacere e seminiamo, il vento
muoverà le orecchie penzolanti e porterà le nostre confessioni altrove.

\begin{quote}
Desidera nasconderle, e per vergogna\textbackslash{}
si prova a coprire le tempie con una benda di porpora.\textbackslash{}
Ma il servitore che aveva il compito di tagliargli i capelli\textbackslash{}
le vide e, non osando svelare quello che aveva visto,\textbackslash{}
ma pure desiderando di farlo e non riuscendo\textbackslash{}
a tacere, si appartò e scavò un buco per terra,\textbackslash{}
e sussurrò a bassa voce alla terra scavata come\textbackslash{}
aveva visto le orecchie del suo padrone;\textbackslash{}
poi ricoprì di terra la sua spiata,\textbackslash{}
ricoprì il buco e se ne andò via in silenzio.\textbackslash{}
Ma in quel punto cominciò a crescere un fitto\textbackslash{}
bosco di canne e quando, dopo un anno, fiorirono,\textbackslash{}
tradirono il seminatore, e mosse dal lieve soffio dell'Austro,\textbackslash{}
riferirono le parole sepolte e denunciarono le orecchie di Mida.
\end{quote}

\subsection{Aumentarne la Forma Immergerne il Contenuto}\hypertarget{aumentarne-la-forma-immergerne-il-contenuto}{}\label{aumentarne-la-forma-immergerne-il-contenuto}

{[}{]}\{\#c:afic label=''c:afic''\}

\begin{quote}
Consapevole come sono che ogni osservazione risente dei tratti
personali dell'osservatore - cioè riflette troppo spesso il suo stato
psicologico piuttosto che quello della realtà osservata - propongo di
accogliere quanto segue con una congrua dose di scetticismo, se non
con incredulità totale. L'unica cosa che l'osservatore può rivendicare
a titolo di giustificazione è che anche lui possiede la sua piccola
quota di realtà, che sarà inferiore per ampiezza, forse, ma in qualità
non ha nulla da invidiare al soggetto considerato\footnote{Iosif Brodskij - \emph{Fuga da Bisanzio}, \emph{Adelphi}, 1987.}.
\end{quote}

Condivido, attraverso le parole di \emph{Josif Brodskij}, l'osservazione,
l'ascolto, i tratti psicologici della mia realtà cercando di
rivendicarne, un passo alla volta, la relativa piccola quota di reltà.

Attraverso un'analisi critica di alcune problematiche inerenti la
produzione e il trattamento del suono in sala da concerto,
l'osservazione porta alla necessaria considerazione di soluzioni
alternative a quelle \emph{dominanti} che, nella musica, provengono
esclusivamente dai relitti di un'industria discografica. Già perché
seppur non ci sia un'evidente similitudine nelle due circostanze
(ascolto domestico verso ascolto concertistico) tecnologicamente e
tecnicamente le affrontiamo con gli stessi presupposti (prevalentemente
sbagliati, in entrambi i casi) che troppo spesso rispondono all'unico
dettame della \emph{soluzione più comune}. Una forma di autodifesa quindi
dalla troppo comune risposta \emph{«\ldots{}perché si fa così!»} o della sua
variante peggiorativa \emph{«\ldots{}perché fanno tutti così!»}. Atteggiamento,
questo, che per semplici dimostrazioni matematiche non può che portare
alla mediocrità.

A suffragio del mio metodo \emph{antagonista} nella lunga battaglia in difesa
delle parole e del loro significato, faccio rierimento ad un classico
della letteratura \emph{antagonista}: le \emph{Etimologie o Origini} di Isidoro di
Siviglia, dove si può apprendere una buona spiegazione della diffferenza
tra \emph{arte} (sala da concerto) e \emph{disciplina} (discografia, o industria).

\begin{quotation}
I. DELLA DISCIPLINA E DELL'ARTE.

{[}1{]} Il termine \emph{disciplina} deriva dal verbo \emph{discere}, che
significa \emph{apprendere}, ed è pertanto sinonimo di \emph{scienza}. Il verbo
\emph{scire}, infatti, che significa \emph{sapere}, viene dallo stesso verbo
\emph{discere} poiché nessuno \emph{scit}, ossia \emph{sa}, se non chi \emph{discit},
ossia \emph{apprende}. Da un altro punto di vista, una \emph{disciplina} è così
definita perché \emph{discitur plena}, ossia perché si \emph{apprende
pienamente}. {[}2{]} L'\emph{arte}, invece, è stata così chiamata perché
fondata \emph{artis praeceptis regulisque}, ossia \emph{su rigorosi precetti e
regole}. Vi è chi dice tale vocabolo sia stato coniato dai Greci per
derivazione ἀπό τῆς ἀρετῆς, cioè \emph{dalla virtù}, cui gli stessi Greci
diedero il nome di scienza. {[}3{]} La differenza fra \emph{arte} e
\emph{disciplina} fu stabilita da Platone ed Aristotele i quali sostennero
che l'\emph{arte} riguarda ciò che può avvenire in modi diferenti, la
\emph{disciplina}, invece, ciò che non può avvenire se non in un unico
modo: quando, infatti, si discute sulla base di argomentazioni certe,
oggetto della discussione sarà una disciplina; quando, invece, si
tratta di un qualcosa di verosimile od opinabile, l'oggetto della
trattazione sarà chiamato arte\footnote{Le sette discipline liberali secondo la catalogazione di Isidoro
di Siviglia sono \emph{grammatica}, \emph{retorica}, \emph{dialettica o logica},
\emph{aritmetica}, \emph{musica}, \emph{geometria}, \emph{astronomia}.}.
\end{quotation}

Le scienze di oggi, discipline che affondano le proprie radici in un
terreno artistico, sono frutto di tecnica e abilità derivate dallo
studio e dall'esperienza. A separarle è il linguaggio nella sua presenza
ripetitiva ed univoca nella disciplina, ineffabile e polisemica
nell'arte.

Luigi Nono, nell'intervista Ágnes Hetényi del 1986, descrive questa
degenerazione ideologica con parole pesanti:

\begin{quote}
{[}\ldots{}{]} in campo musicale scrivono tutti per le orchestre, per le sale
da concerto in modo tradizionale, come se fossimo nell'Ottocento;
anzichè affrontare problemi nuovi, il che vuol dire portare il
disordine necessario all'interno delle organizzazioni tradizionali\ldots{}
\end{quote}

Quando un individuo smette di lavorare su se stesso, in termini di
ricerca, accade perché l'intera società è già soggiogata ad una
fruizione in luogo di una creazione.

\begin{quote}
Si può subire passivamente un'impostazione tecnologica usata da altri,
questi altri {[}\ldots{}{]} possono essere {[}\ldots{}{]} gli industriali italiani,
possono essere quelli che {[}\ldots{}{]} organizzano i programmi per i
computer o per le scuole, possono essere funzionari di partito, come
si dà una notizia.
\end{quote}

C'è quindi un principio, nella prassi musicale attuale, quello che
consiste nell'adozione di soluzioni \emph{standard} anche nel caso di
circostanze \emph{uniche}.

Ogni giorno ascoltiamo l'espressione \emph{«i giovani non sanno più
ascoltare»}. Non sono d'accordo, sono portato ad andare oltre questa
sempliciotta negazione. Un po' perché mi ritengo ancora giovane, e so
ascoltare, un po' perché sono convinto che non può esserci un buon
ascoltatore in presenza di un cattivo messaggio. E mi chiedo quale possa
essere buon messaggio, dei grandi, che questi giovani stanno perdendo.
Un primo approdo di questo tipo di ragionamento è quello di suggerire
che non esiste l'ascolto: un solo ascolto, al singolare; esistono
infiniti ascolti.

Siamo al punto cardine, gli ascolti possibili, le soluzioni adeguate nel
rispetto dell'unicità della circostanza. Abbiamo disciplinato l'ascolto
assecondando le necessità industriali perdendo, rapidamente e solo nel
novecento, il signiicato di alcune parole importanti per la descrizione
delle nostre attività musicali. Abbiamo perso la capacità (in arte) dei
differenti ascolti del fare concertistico (nonostante sia stato l'unico
modo di fare musica per millenni) a vantaggio di un ascolto condizionato
(disciplinato) proveniente dall'industrializzazione dell'ascolto che
lentamente è travasata anche nelle sale da concerto, prima tecnicamente,
attraverso gli oggetti, poi mentalmente, attraverso le idee.

\emph{Cosa può condividere un ascolto fisico, concertistico, con quello di un
prodotto discografico? Abbiamo smesso di esplorare mari e monti perché
gli stessi paesaggi li possiamo vedere in streaming? Siamo ancora
esploratori?}

\begin{quote}
Ogni movimento su una superficie piana che non sia dettato da
necessità fisica è una forma spaziale di autoaffermazione, si tratti
di imperialismo o di turismo.
\end{quote}

Strapperò questa frase dal suo contesto, non ignorando la circostanza in
cui l'autore, \emph{Josif Brodskij}, si paragona a Costantino nel suo
turistico viaggio a Istanbul, e piegandola a parabola perfetta di una
lezione di elettroacustica, di interpretazione, di musica, si può
arrivare a concepire una visione alternativa, non turistica, non
imperialista, fisicamente necessaria, della gestione del suono
riprodotto nel contributo che quersta visione può dare
all'amplificazione o alla proiezione dei suoni nello spazio.

La ricerca musicale ha un ruolo sociale, quanto lo scrivere, quanto rito
del concerto. L'attività musicale è così completa, totale.

Questo breve anatema fa riflettere su una molteplicità di attività
legate alla riproduzione, diffusione e proiezione del suono. Innanzi
tutto fissa un punto cruciale, quello di \emph{necessità fisica}. Il suono è
fisico. È quella parte dell'attività fisica che percepiamo uditivamente
con le orecchie. Il suono come necessità fisica e la necessità di
esplorare il suono. Esplorare, un'attività fondamentale che da sola è
paradigma di ricerca. Un secondo punto fondamentale che rubo al pensiero
di Brodskij quello di \emph{forma spaziale}. Quindi c'è una reale
possibilità, per alcuni una necessità, quella dell'erratico esplorare la
forma spaziale dei suoni. È altrettanto urgente odiare l'idea di essere
indicato come turista del suono, motivo per cui questa non è una ricerca
sull'autoaffermazione, bensì sulla condivisione di un'esperienza dettata
da una necessità fisica di forma spaziale. \emph{Aumentarne la forma}. Ho
scelto aumentare in luogo di amplificare, per dare una prospettiva
diversa alla questione. Con amplificare generalizziamo l'aumentare il
valore di una data grandezza fisica, come fa un amplificatore. E questa
grandezza nel mondo dei suoni riprodotti dagli altoparlanti può essere
solo quella del'ampiezza. In quest'ottica un altoparlante ha il solo
scopo di amplificare il volume di un suono. Aumentare d'altra parte
significati simili, ma divergenti quanto basta per avere altre
prospettive: rendere maggiore, nelle dimensioni o nella quantità.
Dimensioni maggiori. Nell'idea di un suono che abbia una sua forma
acustica, aumentare significa dialtare nello spazio quella forma, in
tutte le dimensioni. Amplificare un solo parametro è quindi un
\emph{movimento su una superficie piana}, di stampo turistico-imperialista.
Si amplifica l'ampiezza, l'intensità ne risulta potenziata. Il concetto
di aumentazione della forma invece punta il dito sulla necessità di
accrescere l'intero complesso geometrico e timbrico con una
concertazione di gesti, multidimensionali e multitemporali.

Per introdurre il concetto di forma del suono mi servirò di una parola
che nel novecento è stata abusata e trasfigurata: \emph{stereo} (vedi
\hyperlink{c:ee}{{[}c:ee{]}}\{reference-type=''ref+label'' reference=''c:ee''\}). Una
concezione di suono solido non può accontentarsi di un'amplificazione di
stampo turistico-imperialista, con pretese di autoaffermazione
attraverso un'imposizione di volume. Dopo tutto più forte non significa
più grande, non soprattutto migliore. L'aumentazione solida ha lo scopo,
semplice e naturalistico, di rendere più grande, multidimensionalmente,
un complesso sonoro. Più grande significa osservabile da lontano, in una
eventuale osservazione prospettica. Più grande significa anche in scala,
parametricamente proporzionale.

Non si può distruggere una cattiva idea di stereofonia senza affondare
il colpo decisivo: un'idea di suono senza luogo non ci è concessa.
Questo nella fisica quanto, e soprattutto, nella musica. Un'idea così
forte come quella di poter ascoltare la musica ovunque ed in qualsiasi
condizione come può essere contrastata? Con l'idea che \emph{nessun luogo è
superfluo} e specificando che anche nella sala da concerto può non
accadere quell'incantesimo di innamoramento per l'attività stessa
dell'ascolto. Il mio stesso innamoramento si è confrontato con la
frustrazione proprio in quei luoghi, ed è li che ho trovato le risposte
e tutte le mie necessità hanno poi trovato conforto. Ed è li che si
compie il secondo gesto magico: \emph{immergerne il contenuto}. Nel rapporto
dialettico tra suono e spazio non c'è, nella nostra percezione,
soluzione di continuità. Questo vuol dir che nel continuo
spazio-temporale del gesto musicale un clarinetto può attaccare un suono
acuto dal nulla e spingerlo molto dolcemente ai livelli udibilima,
attenzione, il suono appare delocalizzato. Non un moto rettilineo su una
superficie piana, ma un errare vagabondo tra la materia che lo fa
comparire alle tue spalle, a destra. In quel momento si può essere solo
due persone: l'ignobile indifferente turista; il consapevole esploratore
con un lungo sorriso stazionario sulla faccia. In questo confronto tra
attività in moto rettilineo e quelle erranti ed erratiche dell'incedere
sperimentale, c'è spazio per il giudizio personale, per l'emozione e la
contemplazione.

Qui si introduce un anello fondamentale nella catena del ragionamento:
non può esserci affinità timbrica tra suoni senza affinità spaziale.
Contemporaneamente: non può esserci manipolazione timbrica senza
un'inevitabile manipolazione spaziale. Timbro e forma sonora come
visioni parametriche del complesso sonoro.

L'amplificazione prevede un parametro ed uno solo

In completa assonanza con l'idea di arte esposta da Isidoro di Siviglia
c'è quella di creatività alla quale Nono fa riferimento per prendere le
distanze dal fatto economico:

\begin{quote}
{[}\ldots{}{]} il rapporto tra individuo, potenzialità da sviluppare al
massimo e qualità di creatività, di creazione\ldots{} Un'altra cosa che
penso è che bisogna cercare finalmente di distinguere il momento
creativo e il momento tecnologia, puramente economico\ldots{} la creatività
ha tutte le porte da spalancare e tutte le porte spalancate {[}\ldots{}{]}
\end{quote}

Il pensiero antagonista è quindi un percorso di riappropriazione,
riconquista del rapporto tra spazio e tempo:

\begin{quote}
Personalmente ritengo che uno dei compiti dell'oggi è non solo di
riscattare, ma di scoprire tutto quello che è stato messo in disparte,
che è stato allontanato\ldots{}perché creava disordine, non rispondeva
{[}\ldots{}{]} a certi principi {[}\ldots{}{]} oppure portava con sé altre esigenze
{[}\ldots{}{]} Per me è molto importante il cercare continuamente,
continuamente cercare e trasformare, sondare, buttare in aria, proprio
il disordine inteso dal filosofo René Thom quando parla della teoria
delle catastrofi. La catastrofe è qualcosa che avviene
improvvisamente, inattesa, e sconvolge le tue categorie mentali, per
cui sei obbligato a trovare un altro mondo, altre metodologie
analitiche di conoscenza. {[}\ldots{}{]} non dare definizioni, non dare
soluzioni, non mostrare obiettivi, ma dare il massimo delle
possibilità informative, cose che poi ciascuno collega o connette
{[}\ldots{}{]}
\end{quote}

L'informazione alla quale mi aggancio per iniziare questa esposizione è
quella offerta da Jean-Claude Risset, il quale, per spiegare il suo
pensiero \emph{«Music is meant to be heard: perception is central in (my)
computer music.»} parte da 1875, anno \emph{catastrofico}, anno in cui
individua due momenti che avranno un ruolo sostanziale nella
modificazione del rapporto tra uomo, suono, tempo, spazio. I fatti sono:
l'introduzione del telefono, e quindi dei principi tecnici secondo cui
fu possibile spedire il suono attraverso un segnale elettrico per lunghe
distanze; l'introduzione della registrazione sonora.

Prima del 1875, per produrre suono, era necessario usare o costruire una
macchina vibrante. Si conosceva la causa ascoltando la qualità del suono
riferita all'origine e, naturalmente, quella causalità dal suono. Con la
registrazione non fu più necessario rompere un vetro per sentire il
suono del vetro rotto. La causalità divenne diversa, e allora fu anche
possibile iniziare a studiare il suono come un oggetto, anche invertito
nel tempo. Fu possibile perché il suono aveva cambiato dominio, cioè era
applicato allo spazio.

\begin{quote}
Ci sono problemi che alle volte anch'io non capisco, nel senso che se
ne porgono continuamnte di nuovi. È da anni che lavoro e sperimento
negli studi di \emph{Live Electronics}di Friburgo, della Sudwestfunk. Si
tratta delle trasformazioni in tempo reale del suono e della voce, e
del comporla con lo spazio, usando le tecnologie di oggi, con i vari
altoparlanti disposti nella sala. C'è qualcosa di nuovo solo sul piano
tecnico, perché se prendiamo la Scuola di S. Marco veneziana di Andrea
e Giovanni Gabrieli, Monteverdi, di Willaert, con le composizioni a
più cori, la grande scuola spagnola all'epoca di Filippo II {[}\ldots{}{]} si
faceva musica per otto organi e quattro cori, cioè si suonava lo
spazio come componente musicale, non come poi la prassi dell'ottocento
usa lo spazio, mettendo dentro l'orchestra e quel che succede succede.
Quindi altri studi, anche studi di fisica architettonica, studi di
processi di eco, di riverberazione, di materiali acustici. {[}\ldots{}{]} Qui
una composizione non è data una volta per sempre, perché per ogni
spazio noi dobbiamo cambiare i programmi dei computer e modificando i
rapporti della trasformazione si modifica anche il rapporto acustico;
{[}\ldots{}{]} il grande fascino di questo per me è veramente la non
ripetitività. {[}\ldots{}{]} Un interprete non deve studiarsi la parte ma
veramente partecipare. {[}\ldots{}{]} Cioè vedi come noi possiamo con la
tecnologia di oggi studiare molto meglio, cioè studiare in un altro
modo.\textbackslash{}
Luigi Nono 1986
\end{quote}

Ascolto come principio, mezzo, fine. Le sfumature acustico
elettroacustico tingono solo il contesto e portano la riflessione al
punto risolutivo: cercare soluzioni mediante la ricerca musicale.

\begin{quote}
L'unica cosa che l'osservatore può rivendicare a titolo di
giustificazione è che anche lui possiede la sua piccola quota di
realtà, che sarà inferiore per ampiezza, forse, ma in qualità non ha
nulla da invidiare al soggetto considerato. A una parvenza di
obiettività si potrebbe arrivare, non c'è dubbio, attraverso
un'autocoscienza completa al momento dell'osservazione. Non credo di
essere capace di tanto; in ogni caso, non era nelle mie aspirazioni.
Comunque spero che qualcosa di simile sia avvenuto. \footnote{Iosif Brodskij - \emph{Fuga da Bisanzio}, \emph{Adelphi}, 1987.}.
\end{quote}

\subsection{Ascoltare la Complessità}\hypertarget{ascoltare-la-complessit}{}\label{ascoltare-la-complessit}

{[}{]}\{\#c:cac label=''c:cac''\}

\emph{Come ascoltiamo la complessità?}

Lo studio della spettromorfologia dei suoni ha radici profonde nella
storia dell'analisi del suono, nella narrazione dei tratti
caratteristici dell'oggetto sonoro, nel suo svolgimento temporale.
Questo workshop pone l'obiettivo di giungere alla possibilità di
integrare quelle conoscenze con tecnologie di ripresa e diffusione
sonora multidimensionale. Ogni fenomeno acustico si sviluppa mediante
una complessa propagazione attraverso lo spazio. Dal punto di vista
dell'ascolto questa complessità combina molteplici elementi, in una
relazione topologica tra forma e spazio, che si manifestano come fonti
inestricabili di informazioni sulla natura dello stesso segnale. Questi
sono gli aspetti più vulnerabili del suono in un concerto
elettroacustico. Allo stesso tempo, sono anche i parametri critici su
cui agire per ottenere risultati sonori di pura magia. Quando le
relazioni spazio temporali tra sorgenti acustiche ed elettroacustiche
sono multidimensionalmente equilibrate e l'ascoltatore proiettato in un
altrove contemporaneo, noi siamo maghi.

Alla profondità di ogni letteratura, il tempo. Considerando
l'ascoltatore come osservatore del tempo, il ruolo della musica come
gioco di memoria, come speculazione del compositore sul rapporto
tempo-relazione attraverso la mente sconosciuta dell'ascoltatore, cosa
succede nello spazio dal gesto esecutivo all'attività di un osservatore?
Sono le preziose relazioni temporali da scoprire e spiegare. La
letteratura elettroacustica spiega cos'è la forma-suono, principalmente
attraverso concettualizzazioni astratte. Tuttavia, essa è realmente
osservabile anche in ambito acustico. È il modo in cui possiamo
immaginare il suono del corno e le sue differenze con un fagotto o un
diverso spazio riverberato intorno a loro. È la memoria plastica della
relazione ritmica, quindi è necessaria una rappresentazione della forma
del suono. Quale approccio all'analisi della forma del suono di un
oggetto acustico è più efficiente? L'efficienza della rappresentazione è
necessaria per mettere a fuoco la relazione tra questi fondamenti
fondamentali e il modo di ascoltare delle persone. Deve essere legata
alla reale possibilità di messa in scena, e deve produrre una soluzione
tangibile per la musica dal vivo. È possibile riprodurre una forma
sonora e confrontare la sua riproduzione elettroacustica con quella
acustica? Sì, è possibile. Il solido tridimensionale più semplice è il
tetraedro. La diffusione del suono attraverso le sue quattro facce
permette di ottenere la riproduzione della forma sonora più precisa e
più efficiente. Qual è l'impatto emotivo della relazione micro-ritmica
tra forme sonore miste durante un concerto dal vivo? È una domanda molto
irrisolta. Si potrebbe rispondere con strategie precise per mappare,
tracciare e analizzare la musica scritta sullo stesso nucleo di dati
osservati. Il palco è la metà dell'attitudine all'ascolto. La ricerca
per cucire entrambi.

La ricerca stabilirà una comunità di pensiero intorno a fatti rilevanti
che emergono dal lavoro profondo sui fondamenti temporali della
percezione.

La musica non è solo composizione. Non è artigianato, né solo un
mestiere. La musica è il pensiero.

Queste parole di Luigi Nono spiegano l'attitudine a fare musica con
l'inesorabilità della percezione, che implica ascoltare e pensare,
scrivere e parlare, e cambiare idee. Queste attività sono il ritmo della
composizione, le topologie di un'idea musicale. La ricerca stabilirà una
comunità di pensiero intorno a fatti rilevanti che emergono dal lavoro
profondo sui fondamenti del tempo della percezione.

Il mondo, nel suo fluire costante ed incessante nel tempo reale della
vita appare estremamente semplice e comprensibile (di una comprensione
mediante osservazione che attinge inevitabilmente alla bisaccia
culturale di appartenenza). La complessità di ciò che percepiamo si può
rivelare solo attraverso un'indagine profonda. Quando per esempio, per
osservare e comprendere, fermiamo il tempo, formuliamo algoritmi,
stabiliamo relazioni. Passa quindi da semplice a complesso in funzione
della volontà di entrare nei labirinti della comprensione. È semplice
tutto ciò che resta fuori dal labirinto, la porta d'accesso, è complesso
tutto ciò che richiede una relazione tempo-culturale attraverso il
labirinto.

Superficialmente, comprendiamo il tempo fino a che non dobbiamo definire
il tempo. E lo comprendiamo meglio solo quando siamo scesi nelle ininite
piege del tempo, il quale ci annuncia che non avevamo capito nulla del
tempo.

Comprendiamo l'ascolto ed i fenomeni acustici. Approfondiamo: \emph{siamo
capaci di descriverne il funzionamento e nel caso migliore replicarlo in
laboratorio?} Si alza un sì planetario. Sappiamo molto di come
ascoltiamo, sappiamo molto del canto di un usignolo; potremmo quindi
rispondere sì, sapremmo replicare tutto il sistema in laboratorio e
desumere che quella capacità rappresenta, come modello, tutto
l'ascoltabile. Ma è un sì di superficie, una semplificazione tipica del
momento culturale. Il che è facilmente dimostrabile con la successiva
domanda: \emph{siamo in grado di descrivere la complessità di una variazione
timbrica e simultaneamente dinamica di un archetto che sfrega le corde
di un violino, dal *nut} al \emph{ponticello}, da \emph{inclinato} a \emph{piatto},
\emph{leggero} e poi \emph{pesante}\ldots{}e lo sappiamo descrivere in un complesso
articolatorio di un suono ogni \$500ms\$; sappiamo poi descrivere questo
comportamento orchestrato in un quartetto d'archi all'interno di uno
spazio riverberante e con il pubblico in grado di muoversi liberamente?*
Ancora si? Lo sappiamo infine riprodurre in laboratorio? O almeno
osservare una pratica efficace per amplificare l'evento acustico in sala
da concerto? Si? Mediante altoparlanti tradizionali che possono suonare
controllati solo sul fronte e solo mediante il parametro dell'ampiezza?

\paragraph{COMPLESSITÀ \{\#complessità .unnumbered\}}\hypertarget{complessit-complessit-unnumbered}{}\label{complessit-complessit-unnumbered}

caratteristica di un sistema (perciò detto complesso), concepito come un
aggregato organico e strutturato di parti tra loro ingerenti, in base
alla quale il comportamento globale del sistema non è immediatamente
riconducibile a quello dei singoli costituenti, dipendendo dal modo in
cui essi interagiscono. - treccani.it

Mi si perdonerà quest'indugio enciclopedico, funzionale solo ad
evidenziare lo stato attuale della ricerca per semplificazione
attualmente attuata di \emph{default}.

\textbf{Sullo sondo di ogni letteratura, il tempo}.

Il tutto finora esposto ci riconduce all'antica questione se siano le
scienze ad alimentare le arti, o se quest'ultime abbiano una certa forma
di libertà che permette salti e collegamenti impossibili nella visione
scientiica. Si può scegliere da che parte stare, tuttavià l'evidenza
sulla quale vorrei puntare il dito sta nel fatto che le articolazioni
musicali di cui sopra sono il mondo semplice di un compositore
contemporaneo (uno: Giorgio Netti, S. Giovanni Rotondo, Puglia) mentre
il mondo scientifico di riferimento (tre: AALTO, Finlandia; Centro
RITMO, Norvegia, UKRI CENTER, Londra) osserva il mondo reale (acustico)
con estrema semplificazione.

Volendo concludere questo argomento con violenza, l'intelligenza
artificiale è in grado di risolvere problematiche complesse laddove lo
sguardo umano è ancora capace di leggere la complessità; nel rapporto
con le arti l'intelligenza artiiciale è allo stato di demenza. È
alimentata da semplici ricette liofilizzate e scolastiche. Per
alimentare un pensiero artistico complesso, al pari di un algoritmo, è
nnecessario uno sguardo complesso alla realtà a cui l'arte appartiene.

\subparagraph{NICCHIA \{\#nicchia .unnumbered\}}\hypertarget{nicchia-nicchia-unnumbered}{}\label{nicchia-nicchia-unnumbered}

La musica contemporanea è spesso definita come un argomento di nicchia.

Cos'è un argomento di nicchia? Un pensiero di nicchia? È un pensiero
comprensibile solo a pochi? È probabile. Ma non lo sono tutti i pensieri
specifici?

Un'equazione di secondo grado appartiene al pensiero globale, così come
la possibilità di un testo in latino. Sono passaggi essenziali della
formazione mentale in età adolescenziale che guarda in nprospettiva al
mondo scientifico. Non l'ascolto e l'analisi di una sequenza di Berio.
Non un quadro di Burri. Così cresciamo, nell'epoca cresciamo la mente
matematica, in cui comprendiamo l'integrale e le derivate, ascoltiamo
merda. Giungiamo all'esame di stato senza sapere chi sia stato Adriano
Olivetti, ignorando che viviamo nel mondo di cui egli stesso è stato
start-up. =======

\emph{Cosa accade alla forma sonora di uno strumento in presenza di tecninche
estese applicate allo strumento?}

Nel 2018 ho avuto la possibilità di ampliicare \emph{ur}, brano per
contrabbasso solo, di Giorgio Netti, con cinnque difusori tetraedrici ed
una microfonazione dedicata. Penso a quei modi, rifletto sulla relazione
tra forma e movimento. Una mano che inizia a produrre suono lì, nello
stesso luogo dello strumento dove pochi minuti prima premeva solo
posizioni, (la mano sinistra di \emph{ur} nel secondo movimento) d'un tratto
diventa esplosione di forma acustica e musicale.

Di nuovo una qualità che presuppone dei contenuti coerenti con delle
caratteristiche specifiche. Ora, nel mondo elettroacustico, del suono
prodotto o riprodotto elettricamente, dovremmo essere in grado di
effettuare lo stesso ragionamento sostituendo alla persona che parla un
diffusore generico. Come per l'essere umano, la voce è esempio di suono
proprietario anche per il diffusore si può scegliere un suono che lo
caratterizzi elettroacusticamente, un suono che lo rende particolare: il
suono definito rumore rosa. Posizionato il diffusore nella stessa stanza
e con le stesse circostanze di ascolto precedenti, \emph{avremmo una
condizione di ascolto stereofonico?} Ovviamente si. Un solo diffusore
può costituire una condizione d'ascolto stereofonica. In questo caso
l'oggetto acustico è un diffusore che esprime se stesso attraverso un
suono non informativo. Un rumore è caratterizzato da un'assenza di
informazione, fatta esclusione del fatto stesso che è rumore. Questa
descirizione di stereofonia possibile anche con un solo soggetto sonoro,
voce o diffusore che sia, non è così comune e condivisa. Ciò accade a
causa del fatto che spesso si fa confusione tra stereofonia, o
stereofonica, come aggettivo applicato alla tecnica di diffusione e
registrazione, piuttosto che alla qualità percettiva che queste tecniche
dovrebbero suggerire. Per arrivare a descrivere la tecnica dobbiamo
percorrere ancora alcuni passi. Nel momento in cui si passa da un
dominio puramente acustico sia esso derivante da una voce umana quanto
un rumore diffuso attraverso un altoparlante, ad un dominio di
riproduzione acustica, ovvero di rappresentazione attraverso meccanismi
e tecniche allora cambia completamente lo scenario acustico e le
circostanze di ascolto. \emph{Un diffusore tradizionale può riprodurre una
voce umana o un diffusore che suona rumore rosa?} Si certo che può
riprodurli. \emph{Questa riproduzione costituirebbe un ascolto stereofonico
della sorgente originale?} No, non lo sarebbe.

\subsection{Verso un pensiero tetraedrico}\hypertarget{verso-un-pensiero-tetraedrico}{}\label{verso-un-pensiero-tetraedrico}

{[}{]}\{\#c:vpto label=''c:vpto''\}

::: flushright
\emph{Conosco la metà di voi solo a metà e\textbackslash{}
nutro per meno della metà di voi\textbackslash{}
metà dell'affetto che meritate.}
:::

\subsubsection{Incipit}\hypertarget{incipit}{}\label{incipit}

\emph{Chi è il Signore degli Anelli?}

È chiaramente una domanda buffa. La mia speranza è che le prime risposte
ottenute, dopo timidi sguardi increduli e incoraggiamenti, siano nomi di
persona: \emph{Sauron!} Oppure \emph{Frodo\ldots{}} poi qualcuno dirà \emph{L'Anello!}

\emph{Esatto!}

\begin{quote}
un anello per domarli,\textbackslash{}
un anello per trovarli,\textbackslash{}
un anello per ghermirli\textbackslash{}
e nel buio incatenarli.
\end{quote}

Qualcuno si chiederà perché io abbia iniziato una lezione di musica, di
ricerca elettroacustica, con Tolkien. Sì, sono partito da lontano, dalla
\emph{Seconda Era di Arda}. Mi sto servendo della \emph{mitopoiesi} di un
capolavoro d'arte per introdurre due questioni da analizzare prima di
concentrarmi su aspetti, per così dire, tecnici. Vorrei dapprima
precisare il mio interesse per \emph{la cosa}, l'oggetto. Ho usato un \emph{anello
del potere, l'unico} per stimolare un gioco attorno ad un oggetto
animato. Rispondendo \emph{Sauron, l'oscuro signore}, si sarebbe
semplicemente collegato un signore ad un altro. Un legame uomo-uomo. Chi
è il Signore? Il Signore. Ma indicando l'anello, a signore, \emph{un anello
per ghermirli\ldots{}} in luogo del suo signore voglio spsotare l'attenzione
sull'idea che le cose, gli oggetti, hanno una loro \emph{vita}, le loro
relazioni, fuori dal nostro controllo, dal nostro potere, fuori dalla
nostra sensibilità e noi con la nostra sensiblità siamo come gli
oggetti: relazioni e processi.

\begin{quote}
L'errore è assumere che la fisica sia la descrizione delle cose in
terza persona. È il contrario: la prospettiva relazionale mostra che
la fisica è sempre descrizione della realtà in prima persona, da una
prospettiva. Qualunque descrizione è implicitamente dall'interno del
mondo, da un punto di vista associato a un sistema fisico.
\end{quote}

\subsubsection{Processo}\hypertarget{processo}{}\label{processo}

\begin{quote}
\ldots{}la base che abbiamo per comprendere il mondo è la nostra
informazione sul mondo. che è una correlazione, di cui ci serviamo,
fra noi e il mondo.
\end{quote}

Nella mia formazione (musicale) la parola \emph{oggetto} ha subito diversi
stadi evolutivi. L'\emph{Oggetto Sonoro} della scuola francese è apparso
nella mia vita ventitré anni fa. Non avevo mai sentito parlare di
\emph{oggetto sonoro} prima di incontrare Giorgio Nottoli.

Col tempo mi sono staccato da quel modo plastico di parlare di suono. Il
suono è nelle nostre orecchie. Non è in aria. In aria ci sono le
vibrazioni acustiche, che noi chiamiamo suono se queste ci sfiorano, a
cui siamo sensibili, di una sensibilità tattile ed emotiva, come molte
altre sensibilità umane. Tuttavia, seppur la mia concezione di suono
prende le da una scuola oggettistica, oggettiva, di oggetti sonori
classificabili separatamente, mi viene spontaneo anche ridefinire un
contatto con l'oggetto della mia sensibilità. E questo ho imparato a
farlo da Foucault: \emph{il discorso è una cosa, ha il suo spessore.}

\section{Ricerca}\hypertarget{ricerca}{}\label{ricerca}

\subsection{S.T.ONE}\hypertarget{stone}{}\label{stone}

::: flushright
\emph{Si dice che i compositori abbiano orecchio per la musica e\textbackslash{}
di solito significa che non sentono nulla che arrivi alle loro
orecchie.\textbackslash{}
Le loro orecchie sono murate dai suoni di loro creazione.}\textbackslash{}
John Cage - \emph{45' for a Speaker} (1954)
:::

\begin{quote}
Si dice che la disciplina geometrica sia stata creata dagli Egizi in
occasione di un'inondazione del Nilo poiché, essendo state coperte di
fango le pèroprietà di ognuno, per la prima volta si effettuò la
dovuta ripartizione della terra attraverso l'uso di linee e misure: da
qui il nome di quest'arte che, perfezionatasi in seguito grazie
all'acume dei dotti, permette di misurare gli spazi del mare, del
cielo e dell'universo. {[}\ldots{}{]} La teoria di tale disciplina comprende
le linee, gli intervalli, le grandezze e le figure, e, nelle figure,
le dimensioni e i numeri.
\end{quote}

La letteratura di \emph{Michael Gerzon}  è estremamente pulita e
chiarificatrice su una molteplicità di questioni di ripresa, produzione
e diffusione sonora elettroacustica quanto di percezione. Le sue teorie
e descrizioni sono un percorso verso la comprensione di molti fattori
che attualmente, nonostante l'avanzamento tecnologico, sono lontani e
spesso sconosciuti. Ripartire dalle sue prime ricerche è necessario per
comprendere l'efficacia del pensiero tetraedrico.

Nel 1970 Gerzon cerca di spiegare come si possa avere una buona
riproduzione quadrifonica da una sorgente stereofonica a due canali.

\begin{quote}
It is generally accepted that three speakers are not adequate for good
surround sound, due to the limited listening area and the wide angle
between the loudspeakers. This has led many people to assume that,
because four loudspeakers are necessary for surround sound, therefore
one needs to record four channels. The author has shown1 that even
two-channel recordings can be made to give a genuine surround sound
(albeit with some defects), and this suggests that three channels
might be quite sufficient to convey all the information required for
quadraphonic reproduction.
\end{quote}

\subsubsection{Spherical Tetrahedral ONE}\hypertarget{spherical-tetrahedral-one}{}\label{spherical-tetrahedral-one}

\begin{quote}
{[}I{]} Senza la musica, quindi, nessuna disciplina può considerarsi
perfetta: di fatto, senza la musica nulla esiste. {[}\ldots{}{]} {[}2{]} La
musica muove le volontà trasformando la natura della percezione.
\end{quote}

Anche se l'oggetto finale mi smentisce completamente, non ho mai voluto
progettare un altoparlante. La mia ambizione era superare un certo
limite, che lentamente ho identificato appartenere intrinsecamente
all'oggetto altoparlante. Lentamente, nel corso degli ultimi anni, sono
passato dalla descrizione del limite, alla speculazione compositiva su
esso ed alla successiva realizzazione dell'oggetto altoparlante. Ma la
mia ambizione di partenza era il limite, non l'oggetto. L'oggetto, in
quanto tale è stato il mezzo per giungere a quel limite e superarlo.

Il Limite. In un contesto di musica elettroacustica, ovvero di musica
che si avvale parimenti di oggetti acustici, strumenti tradizionali,
oggetti elettrici ed elettronici, la diffusione riprodotta dei suoni
(mediante altoparlanti) ha sempre rappresentato un tema cruciale,
centrale per l'equilibrio acustico e musicale. Una buona integrazione
tra suoni acustici e suoni elettroacustici può tenere alta l'illusione
di un unicum (ammesso che esso sia l'obiettivo) ma è anche il varco
attraverso cui introdurre il tarlo che farà crollare tutta la
costruzione. Questo ruolo di instabile funzionalità è si parametrabile
all'ingegno che lo regola e lo dispone, ma è anche dovuto al suo più
grande limite: la direzionalità.

Il progetto S.T.ONE nasce dall'esigenza di poter eseguire musica
elettroacustica, sfruttando lo spazio sonoro creato dal mezzo
elettronico, con caratteristiche percettive acustiche analoghe a quelle
degli strumenti tradizionali.

Un altoparlante tradizionale può essere controllato nella sola
dimensione dinamica della potenza e agendo su essa si può cercare un
equilibrio con gli strumenti acustici. Ma gli strumenti acustici hanno
un comportamento molto più complesso, che implica relazioni tra il
fattore dinamico e quello timbrico e sopratutto spaziale del suono,
creando uno scollamento inevitabile tra ascolto acustico ed
elettroacustico. S.T.ONE è il frutto di una ricerca mirata alla
soluzione di questo problema permettendo performance live in cui la
fusione tra i due soggetti, acustico ed elettroacustico, è totale,
poli-dimensionale e completamente nuova all'ascoltatore.

Con il diffusore S.T.ONE si può controllare la propagazione del suono
riprodotto in tutte le direzioni dello spazio e permettere di integrare
questo controllo nei parametri della composizione elettroacustica in un
rapporto dialettico con lo strumento acustico.

Un altoparlante nel migliore dei casi è un ottimo riproduttore di timbro
e dinamica. Ma l'elettroacustica ci ha insegnato che i parametri in
gioco nella descrizione e produzione di suono sono anche altri. La
direzionalità, se vuole essere un parametro descrittivo, deve poter
variare nel tempo. L'unico modo per variare la direzionalità con
l'altoparlante tradizionale è attraverso la moltiplicazione \footnote{Al momento attuale la \emph{wave field synthesis} rappresenta la
massima evoluzione tecnologica di diffusione spaziale mediante
l'utilizzo di molteplici altoparlanti ed anche l'esperienza di
ascolto elettroacustico di maggior effetto a cui ho assistito, nel
2009 presso la \emph{Casa del Suono} di Parma.}.

Per muovere i suoni nello spazio quindi sono necessari diversi
altoparlanti (identici). E va bene, ma che succede se lo spazio di cui
si sta parlando non è rappresentato dal movimento nel tempo, ma dalla
forma? Si ha chiaro il concetto di forma sonora di un oggetto. Un
pianoforte e un toy piano pur condividendo una lista di punti comuni non
producono la stessa forma sonora. Lo stesso può valere per un violino e
un violoncello.

La forma sonora di un oggetto acustico passa molto spesso inosservata,
vissuta come forma intrinseca alle strutture che la generano, spesso non
assume ruolo di protagonista. Ma basta ascoltare il suono diretto di uno
strumento musicale e la sua riproduzione immediata attraverso un
altoparlante tradizionale per capire che la registrazione (tradizionale)
e la diffusione (tradizionale) hanno completamente perso la forma sonora
del complesso strumentale in esame.

\subsubsection{Progettazione}\hypertarget{progettazione}{}\label{progettazione}

Le tecniche di produzione sonora alla base dell'ambisonic forniscono,
soprattutto negli scritti di Gerzon, una importante dimostrazione di
come l'approccio pratico e sperimentale nella ricerca di un metodologia
corretta, sia indispensabile. La sperimentazione avvenuta nel
Conservatorio S. Cecilia di Roma, dal 2008 al 2013, sotto la guida di
Piero Schiavoni, è stata indispensabile per raggiungere un grado di
conoscenza e maturazione tale da poter giudicare una tecnica piuttosto
che un altra. Le molteplici sessioni di registrazione comparata, sia
stereo che surround, le sessioni di ascolto comparato, le visite presso
luoghi chiave come la Casa del suono di Parma e la sede del costruttore
Zingali, sono state tutte esperienze necessarie alla maturazione di una
identità collettiva condivisa. Da qui si parte. Senza queste premesse
mancherebbe un solido substrato di esperienze essenziali alla
comprensione del contesto lavorativo.

Si parte dalle teorie di Gerzon, e dalle sue esperienze di ascolto.
Gerzon identifica in alcune metodologie di ascolto l'incapacità di
restituire fedeltà al campo sonoro originario. Identifica un limite, ed
il suo processo di superamento di quel limite è a dir poco visionario.
Riscrive da zero un metodologie di registrazione e la fornisce di
formule matematiche che descrivono un sistema chiaro e puntuale.
Descrive nella forma del tetraedro la figura semplificata di un sistema
tridimensionale complesso. Ma il tetraedro di Gerzon è un tetraedro di
descrizione, di analisi del campo sonoro naturale. È la metodologia di
registrazione che pone alla base di tutte le sue teorie ma, nel momento
della diffusione, fa uso di normali altoparlanti hi-fi disposti intorno
all'ascoltatore. Un punto di altro partenza da cui le sue teorie si sono
poi sviluppate. Gerzon stesso ha continuato a scrivere ed evolvere il
proprio sistema. Ma gli altoparlanti sono rimasti gli stessi ed il
criterio di diffusione il medesimo, concentrico attorno all'ascoltatore.
E l'ascolto di un altoparlante in questo modo, per quanto esso di buona
qualità e fattura, non può rappresentare completamente una sorgente
acustica naturale.

L'altoparlante che si descrive in questo articolo nasce proprio da qui,
dall'esigenza di unire quelle concezioni tecnologiche illuminate da
Gerzon con la possibilità di usufruire di un ascolto più naturale, meno
caratteristico ed invasivo degli altoparlanti tradizionali. La forma del
Tetraedro è stato il passo naturale vista la quantità di materiale
studiato nel contesto \emph{ambisonic}. Il tetraedro non offre facce
parallele, non ha angoli perpendicolari e non offre fronte diretta
all'osservatore. È il primo dei solidi platonici ed è anche di facile
costruzione.

Parlando del tetraedro si è soliti raccontare un aneddoto. Pare che
Einstein, in una delle tipiche situazioni in cui doveva dimostrare la
sua bizzarra teoria delle quattro dimensioni a degli scettici amici di
scienza, gli consegnò sei stuzzicadenti con il compito di costruirci
quattro triangoli equilateri. Quando nessuno dei presenti, ostinati a
trovare la soluzione del collegamento su di un piano, riuscì a risolvere
il problema, Einstein compose il tetraedro e disse:

\begin{quote}
{[}\ldots{}non sapete usare la terza dimensione, che vivete tutti i giorni,
come sperate di comprendere la quarta?{]}\{.sans-serif\}
\end{quote}

Nel novembre 2013 è stato discusso il progetto con il M. Michelangelo
Lupone ne sono derivate alcune importanti conclusioni riguardanti la
cubatura e, soprattuto, la scelta dei componenti diffusori. Il cono
attualmente montato sul tetraedro è doppio cono da 5 pollici\footnote{Sica 5'' 120W dual cone - 5 D 1 CS \$8\textbackslash{}Omega\$}. Il
cono doppio permette una migliore direzionalità della singola faccia e
garantisce una salita in frequenza con meno perdite, soprattutto in
visione del fatto che l'intero complesso non avrebbe avuto componenti
dedicati alle singole bande di frequenza. Diffusori fullrange quindi e
di diametro contenuto per agevolare le frequenze medio acute, quelle che
realmente producono informazione direzionale.

Non solo la costruzione artigianale, ma anche lo sviluppo di tecniche
specifiche di produzione sonora hanno richiesto il loro tempo di
sedimentazione e ragionamento. Il primo brano composto per
l'altoparlante risale al 2010, quindi in un periodo in cui non esisteva
ancora l'oggetto ma la speculazione sui suoi impieghi e sulle sue leggi
funzionali erano già sviluppate. Ma la composizione acusmatica è
semplificata dalla conoscenza diretta delle problematiche di diffusione
e dalla possibilità di sintetizzare un modello compatibile con
l'altoparlante. La sfida è stata trovare un sistema di registrazione
tale da poter portare dentro S.T.ONE un intero sistema acustico
complesso, completo di tutte le sue caratteristiche e forma spaziale.

La possibilità di esporre lo strumento durante l'EMUfest è stata per me
di grande stimolo. Nella mia mente erano chiare le caratteristiche
acustiche e musicali dello strumento, ma non era affatto semplice
scegliere una forma musicale per proporlo al pubblico. Così ricordai di
avere una partitura di Alvin Lucier per due Flauti o Flauto e Flauto
registrato.

\subsubsection{Repertorio Tetraedrico d'oggi}\hypertarget{repertorio-tetraedrico-doggi}{}\label{repertorio-tetraedrico-doggi}

Una delle meraviglie di questa ricerca è stato vedere crescere nelle
persone vicine l'interesse per questi strumenti di pensiero. Pasquale
Citera, primo, ed a lungo unico, interlocutore per i miei ragionamenti
ad alta voce, è stato il primo ad ibridare la sua musica con i miei
strumenti. Nelle nostre chiacchierate ci fu un momento in cui parlavamo
di S.T.ONE ma non esiste ancora uno S.T.ONE. Questo dato temporale
indica che ad un certo punto per le nostre utopie musicali era
necessario uno strumento nuovo, un veicolo d'informazioni non ancora a
noi disponibile.

Di seguito una lista di opere che utilizzano la tecnologia ST.

2010. \emph{Attraverso la lente}

Il legame tra pensiero musicale e progettazione elettroacustica del
sistema tetraedrico risale al 2009, quando stavo lavorando ad
\emph{Attraverso la lente}, un brano acusmatico. Con \emph{All} provavo a muovere
il progetto elettroacustico di un brano e del suo mezzo di diffusione di
pari passo ma a breve mi resi conto che il limite tecnico e costruttivo
per realizzare l'oggetto era molto più vicino di quello compositivo. IL
brano, quadrifonico\ldots{} 

2013. \emph{A. Sax.}

Passati tre anni da \emph{All} il percorso verso S.T.ONE non solo non era
completato, ma nemmeno avanzato di qualche passo. \emph{A. Sax.} doveva fare
ancora i conti con un pensiero musicale più sviluppato dei mezzi a mia
disposizione. \ldots{} 

2014. \emph{13 Degrees of darkness}

\emph{13 Degrees of Darkness} è un brano di Alvin Lucier per due flauti o
flauto e flauto pre-registrato.

2015. \emph{Al nulla di cui essere felici}

Brano

2015. \emph{PS: SONG\#01}

{[}A cura di Pasquale Citera{]}

2016. \emph{PHASOR}

brano

2016. S4FE

brano

2016. \emph{Appunti sul marmo}

{[}A cura di Marco De Martino{]}

2017.\textbackslash{}
\emph{Come ascoltare i dormiveglia delle vedove}

{[}A cura di Pasquale Citera{]}

2017. Cartografie al margine

{[}A cura di Marco De Martino{]}

2017. Les Adieux!

\subsection{STON3L}\hypertarget{ston3l}{}\label{ston3l}

\subsection{STOOGE}\hypertarget{stooge}{}\label{stooge}

\subsection{TETRAREC}\hypertarget{tetrarec}{}\label{tetrarec}

\subsubsection{Presa di posizione}\hypertarget{presa-di-posizione}{}\label{presa-di-posizione}

\emph{Il sistema è corretto, ma va rovesciato.}

La registrazione tetraedrica è ampiamente conosciuta ed alla base del
sistema ambisonic di \emph{Michael Gerzon}. Gerzon la propone nel 1971\footnote{{[}@mg:extetra01{]}} e
su di essa si basano brevetti di microfoni tetraedrici. Il principio
della registrazione tetraedrica di Gerzon è quello di registrare un
campo sonoro attraverso una configurazione microfonica coincidente,
ovvero capsule microfoniche molto vicine tra loro, disposte sulle
quattro facce di un ipotetico tetraedro. In questo modo si possono
descrivere informazioni spaziali provenienti da onde sonore che
attraversano il sistema di capsule. Per chiudere il cerchio, nello
stesso articolo\footnote{{[}@mg:extetra01{]}} Gerzon illustra anche un sistema di ascolto
compatibile con la tecnica di registrazione appena descritta. Il sistema
prevede degli altoparlanti posizionati sui vertici di un ipotetico
tetraedro attorno alla posizione centrale dell'ascoltatore.

Sugli scritti di Gerzon si è svolta la mia riflessione più lunga. Il suo
approccio alla registrazione, ai perché e ai come di alcune questioni
estremamente importanti della ripresa acustica, hanno suggerito l'idea
di una configurazione microfonica come punto di descrizione modulabile
di un evento acustico, in accordo, concertato, con un sistema di
diffusione. Le pratiche descritte da \emph{Michael Gerzon}sono di estremo
interesse e l'aver ripercorso passo dopo passo le sue sperimentazionni
d'ascolto mi a hanno permesso di giungere a Terarec, che quindi nasce
proprio lavorando d'immaginazione intorno alle sue ricerche.

Osservando la catena elettroacustica, la forma di ogni anello di questa
catena, emergono delle questioni: il sistema di diffusione attualmente
in uso, l'altoparlante, è un oggetto terribilmente imbarazzante;
l'ascoltatore rappresenta un punto, collegato ad un altro punto, la
sorgente, attraverso un complesso spazio-tempo; lo strumento quindi è un
punto; l'attuale sistema di diffusione non è per punti, ma per campi. È
necessario ribaltare il sistema. Immaginiamo uno strumento acustico, un
violoncello, questo diffonde i suoni da un punto, potremmo identificarlo
come il suo baricentro sonoro, verso l'esterno, seguendo molteplici
direzioni e regole acustiche che ne determinano la sua forma sonora. È
il primo anello della nostra catena. Il secondo anello è rappresentato
dall'oggetto posto a descrivere questa propagazione sonora, il
microfono. Ora, se intendiamo per microfono un solo diaframma,
percepiremmo ampiezza, altezza, durata, timbro, prossimità, diffusione,
coerenza, ma non direzione. Non avremo alcuna informazione sulla forma
acustica dello strumento.

Il microfono, o a questo punto dovremo dire la configurazione
microfonica scelta, esprime le nostre intenzioni di descrizione
dell'evento acustico. La scelta, quindi, è manifesto d'intenzioni. Il
concetto di catena chiarisce che non ci può essere un anello debole
altresì ogni anello è cruciale.

Il sistema è corretto, ma va rovesciato. L'ascoltatore al centro di un
sistema di diffusione è l'aberrazione introdotta dal nostro mondo
riprodotto. L'ascoltatore nella sua vita reale di ascoltatore non è mai
al centro.

La realtà che circonda un ascoltatore è complessa e ne ignora la
posizione. Un martello pneumatico vibra lungo la via e quello che giunge
alle orecchie del passante è un punto da cui si sprigiona una potenza al
limite del sopportabile. Un flauto che suona in un'aula di conservatorio
è un oggetto sonoro complesso intorno a cui si può camminare ed
apprezzarne i modi propagatori.

Rovesciare un sistema rovesciato. Un altoparlante tetraedrico si
posiziona non più all'esterno di un sistema di diffusione concentrico ma
in un qualsiasi punto dello spazio e da questo punto riproduce suoni in
tutte le direzioni dello spazio. Come riprodurre un flauto attraverso
questo sistema? Rovesciando l'altro sistema, quello di registrazione. Il
flauto non suona più davanti ad un sistema di registrazione più o meno
complesso, più o meno surround. Il flauto suona dentro il sistema di
registrazione. Dentro il Tetraedro. Questa è la nuova procedura che può
portare la registrazione alla rappresentazione multidimensionale della
realtà acustica. Avvolgendo il flauto con un tetraedro di microfoni si
può registrare la sua forma spaziale. Ciò significa che l'effetto
risultante non può non tenere conto del complesso strumento,
strumentista. Le vibrazioni che scorrono lungo il tubo si irradiano in
tutte le direzioni, non soltanto quelle parallele o perpendicolari al
tubo. Ciò significa che se ci si mette dietro al flautista si scorge un
suono di flauto, ma filtrato dall'ombra del corpo dello strumentista. Il
flauto è uno strumento complesso ha un tubo lungo che va sorretto quasi
parallelo al terreno, ed ecco che anche il foro di uscita e la distanza
dalla boccola fungono da variabili alla forma spaziale del flauto. Forma
che si sarebbe completamente persa con una registrazione tradizionale,
ma che può essere facilmente descritta dal complesso tetraedrico
spaziato di registrazione e poi riprodotta da S.T.ONE in maniera
sferica, nel modo più naturale possibile.

\subsection{S.T.OSS}\hypertarget{stoss}{}\label{stoss}

::: flushright
\emph{La fatalità che sembra dominare la storia è appunto l'apparenza
illusoria di questa indifferenza,\textbackslash{}
di questo assenteismo. {[}\ldots{}{]} i destini di un'epoca sono manipolati a
seconda delle visioni ristrette,\textbackslash{}
degli scopi immediati di piccoli gruppi attivi, {[}\ldots{}{]} e allora sembra
che la fatalità travolga tutto e tutti,\textbackslash{}
che la storia non sia che un fenomeno naturale, un'eruzione, un
terremoto, del quale rimangono vittima tutti}\textbackslash{}
Antonio Gramsci - \emph{L'indifferenza} ()
:::

Una registrazione tridimensionale può essere fatta in diversi modi. Una
lista non necessariamente completa:

\begin{itemize}
\item{} A-Format
\item{} B-Format
\item{} qualcosa schoeps
\end{itemize}

Michael Gerzon e Peter Fellgett descrissero i principi teorici e tecnici
dell'\emph{ambisonic} a metà degli anni settanta. Volendo evidenziare gli
aspetti lessicali che ci permettono di riacquisire prospettiva di fronte
ad un uso di terminologia inadatta fondata sull'indifferenza, una delle
osservazioni più acute che vale la pena riportare è di Peter Fellgett:

\begin{quote}
Sound-field microphone. This is an omnidirectional microphone in the
true sense, which is the \emph{opposite} of non-directional; it
characterises in a symmetrical manner the waveform and directionality
of sound arriving from any direction (including vertical
components).\footnote{Peter Fellgett, \emph{Ambisonics. Part one: General system
description}.}
\end{quote}

È un passo importante per me. Riconoscere che chiamiamo qualcosa col
nome sbagliato è un passo importante. Che questo non abbia avuto effetto
commerciale o sociale, nemmeno nella ristretta cerchia di individui che
si occupa quotidianamente di questi argomenti, è un fatto ancora più
rilevante.

\begin{quote}
Se qualcosa è ancora inesplicabile, ciò è dovuto solamente alla nostra
incompletezza conoscitiva, all'ancora non raggiunta perfezione
intellettuale. E ciò può renderci più umili, più modesti, non già
buttarci in braccio alla religione. {[}\ldots{}{]} Perché il passato noi lo
sentiamo bensì vivificare la nostra lotta, ma domato, servo e non
padrone, illuminato re e non aduggiatore. {[}La storia - p 72{]}
\end{quote}

Oggi la comunità elettroacustica anche nel dominio industriale della
musica e del cinema si sta lentamente affezionando alla tecnologia
ambisonic. Nonostante ciò si parla di tridimensionalità come di una
circostanza ``rara''.

Chi mi conosce lavorativamente sa che il mio microfono di riferimento è
il Soundfield, per qualsiasi utilizzo. A lui, secondo solo a lui, alla
Fellgett, il non-direzionale. Un microfono non-direzionale rappresenta
attualmente l'oggetto di ripresa sonora più neutro di cui disponiamo.
Non c'è una ragione per cui dovrei preferire un microfono direzionale ad
uno non-direzionale se non l'indisponibilità di questi ultimi.

I migliori microfoni non-direzionali li fa la DPA. Non c'è una cosa che
non prenderei attraverso un DPA.

\begin{quote}
As a general rule it can be said that if we place a cardioid at a
distance of 17 cm to the source, then an omni placed at 10 cm gives
the same ratio of direct and indirect sound as the cardioid.
\end{quote}

sui microfoni coincidenti e spaziati

\begin{quotation}
The Coincident techniques (localisation cues based only on level
differences between signals) can create proper localisation accuracy,
but will lack envelopment and have a small sweet spot (in two
dimensions - left/right and front/rear!) The advantage of a coincident
array is, that it is compact and portable.

A Spaced microphone surround array will create a three-dimensional
enveloping sensation by providing adequate amount of decorrelation
between the signals (localisation cues are based on time-of-arrival
differences). When adapting the microphone placement (distance and
angle) to the sound field, spaced arrays still provide proper
localisation accuracy.

The Spaced techniques in general give a nice and large sweet area and
you sense the enlarged and enveloping sound stage in a larger
listening field. The disadvantage is the size and in some situations
set-up time. Coincident techniques (localization cues based only on
level differences between signals) can create proper localization
accuracy, but lack envelopment and result in a small sweet spot. The
advantage of a coincident array is that it is compact and portable. A
spaced surround array creates a three-dimensional enveloping sensation
by providing adequate amounts of decorrelation between the signals.
Localization cues are based on differences in time-of-arrival. When
adapting the microphone placement (distance and angle) to the sound
field, spaced arrays still provide good localization accuracy.

Spaced techniques give a large sweet spot area and you sense the
enlarged and enveloping sound stage in a larger listening field. The
disadvantage is the size, visibility and set-up time.

There are some different technologies available to make a surround
microphone work. Given the precise specification that it should be as
compact as possible --- preferably within the dimensions of this
magazine --- then the use of widely spaced mics with decorrelated
signals is, of course, not the way to go. The directionality has to be
obtained mainly by level differences and some head-related spectral
cues. However, the d:mension™ 5100 is not a head-related phase cue
solution, with microphones flush-mounted on a dummy head since that
technology works best when played back on headphones (eliminating
crosstalk and room tone by playing direct-to-ear).

Our initial working model was cut from a cardboard box and had five
omni microphones. The performance was surprisingly good. Envelopment
and localization was rich but frequency response suffered from
comb-filtering. We chose the well-known method of separating the omni
microphones with some absorbing material and the challenge was then to
find a material close enough to not let sound through and yet open
enough not to reflect it. A fiber-like material was developed to act
as acoustic baffles (partition walls) between the microphones like a
Jäcklin Disc does for improving localization in AB stereo arrays.

****

Spaced microphone stereo techniques using an acoustic absorbent baffle

Baffled stereo is a generic term for a lot of different stereo
techniques using an acoustic baffle to enhance the channel separation
of the stereo signals. When placed between the two microphones in a
spaced stereo set-up like A-B stereo, ORTF stereo, DIN stereo or NOS
stereo, the shadow effect from the baffle will have a positive
influence on the attenuation of off-axis sound sources and thereby
enhancing the channel separation. Baffles should be made from an
acoustic absorbent and non-reflective material to prevent any
reflections on the surface of the baffle to cause coloring of the
audio. One of the more well known baffled stereo principles is the so
called Jecklin Disc developed by the Swiss sound engineer Jürg
Jecklin. This technique uses two d:dicateTM 4003 Omnidirectional
Microphones, 130 V or d:dicateTM 4006A Omnidirectional Microphones
spaced 17.5 cm and a special acoustic treated disc with a diameter of
35 cm placed between the microphones. At present there is not an
acoustic baffle available from DPA Microphones.

****

A Jecklin disk is a sound-absorbing disk placed between two
microphones to create an acoustic "shadow" from one microphone to
the other. The resulting two signals can possibly produce a pleasing
stereo effect. A matching pair of small-diaphragm omnidirectional
microphones is always used with a Jecklin disk.

The technique was invented by Jürg Jecklin, the former chief sound
engineer of Swiss Radio now teaching at the University for Music and
Performing Arts in Vienna. He referred to the technique as an
"Optimal Stereo Signal" (OSS). In the beginning Jecklin used
omnidirectional microphones on either side of a 30 cm (1 ft.) disk
about 2 cm (3/4") thick, which had a muffling layer of soft plastic
foam or wool fleece on each side. The capsules of the microphones were
above the surface of the disc, just in the center, 16.5 centimeters
(6½") apart from each other and each pointing 20 degrees outside.
Jecklin found the 16.5 cm (6½") ear spacing between the microphones
too narrow. In his own paper, he notes that the disk has to be 35 cm
(13¾") in diameter and the distance between the microphones should be
36 cm (14 3/16"). The concept is to make use of the baffle to
recreate some of the frequency-response, time and amplitude variations
human listeners experience, but in such a way that the recording also
produces a useful stereo image through loudspeakers. Conventional
binaural or dummy head recordings are not as convincing when played
back over speakers; headphone playback is needed.

The Jecklin disk is a refinement of the baffled microphone technique
for stereo initially described by Alan Blumlein in his 1931 patent on
binaural sound.

There is a noteworthy change from the original small version: Instead
of 30 cm, the disk now has a slightly larger diameter of 35 cm. But
what stands out to an even greater degree, is the greatly enlarged
microphone spacing -- rather than formerly 16.5 cm as a human "head
diameter" (ear distance) there is now a distance of 36 cm
(double-headed?). Jecklin's German from his script: "Zwei
Kugelmikrofone sind mit einem gegenseitigen Abstand von 36 cm
angeordnet und durch eine mit Schaumstoff belegte Scheibe von 35 cm
Durchmesser akustisch getrennt."{[}1{]} Translated: Two omnidirectional
microphones are placed with a distance between them of 36 cm (14
3/16"), and acoustically separated by a foam-covered disk having a
diameter of 35 cm (13¾"). That shows a great difference to the
initial smaller Jecklin Disk of 30 centimeters diameter and the
distance between the microphones of 16.5 centimeters.
\end{quotation}

The acoustic separation of the baffle disk results in level, time, and
frequency response that are called spectral differences.

\section{Strumenti di pensiero}\hypertarget{strumenti-di-pensiero}{}\label{strumenti-di-pensiero}

\subsection{SRUMENTI}\hypertarget{srumenti}{}\label{srumenti}

\subsubsection{TEMPO}\hypertarget{tempo}{}\label{tempo}

\subsubsection{EGBAF}\hypertarget{egbaf}{}\label{egbaf}

\subsubsection{EBBAF}\hypertarget{ebbaf}{}\label{ebbaf}



\end{document}
